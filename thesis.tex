\documentclass[final,oneside]{vutinfth}

% Load packages to allow in- and output of non-ASCII characters.
\usepackage{lmodern}        % Use an extension of the original Computer Modern font to minimize the use of bitmapped letters.
\usepackage[T1]{fontenc}    % Determines font encoding of the output. Font packages have to be included before this line.
\usepackage[utf8]{inputenc} % Determines encoding of the input. All input files have to use UTF8 encoding.

% Extended LaTeX functionality is enables by including packages with \usepackage{...}.
\usepackage[inline]{enumitem} % User control over the layout of lists (itemize, enumerate, description).
\usepackage{multirow}   % Allows table elements to span several rows.
\usepackage{booktabs}   % Improves the typesetting of tables.
\usepackage{subcaption} % Allows the use of subfigures and enables their referencing.
\usepackage{minted}
\usepackage[usenames,dvipsnames,table]{xcolor} % Allows the definition and use of colors. This package has to be included before tikz.
\usepackage{nag}       % Issues warnings when best practices in writing LaTeX documents are violated.
\usepackage{todonotes} % Provides tooltip-like todo notes.
\usepackage{hyperref}  % Enables hyperlinking in the electronic document version. This package has to be included second to last.
\usepackage[acronym,toc]{glossaries} % Enables the generation of glossaries and lists of acronyms. This package has to be included last.
\usepackage{pgfplotstable} % For importing CSV files
\usepackage{pgfplots} % For creating plots
\usepackage{listings, listings-rust, listings-json, listings-url} % Code
\usepackage{hyperref}
\lstset{
    basicstyle=\ttfamily,
    keywordstyle=\color{blue},
    commentstyle=\color{gray},
    stringstyle=\color{red},
    breaklines=true
}

% Define convenience functions here:
\newcommand{\authorname}{Jannis Adamek}
\newcommand{\thesistitle}{Scalability and Event Ordering in PermaplanT's Collaborative Garden Editor}
\newcommand{\thesistitlegerman}{Skalierbarkeit und Ereignisreihenfolge in PermaplanT's kollaborativem Garten-Editor}

\newcommand{\rustsnippet}[1]{%
    \lstinline[
        language=rust,
        breaklines=true,
        breakatwhitespace=true
    ]{#1}%
}
\newcommand{\bashsnippet}[1]{%
    \lstinline[
        language=bash,
        breaklines=true,
        breakatwhitespace=true
    ]{#1}%
}
\newcommand{\pythonsnippet}[1]{%
    \lstinline[
        language=python,
        breaklines=true,
        breakatwhitespace=true
    ]{#1}%
}
\newcommand{\urlsnippet}[1]{%
    \lstinline[
        language=url,
        breaklines=true,
        breakatwhitespace=true
    ]{#1}%
}

\newcommand{\precision}{2}
\newcommand{\perfTableAndGraphFromCsv}[1]{%
    % Load the CSV data into a macro
    \pgfplotstableread[col sep=semicolon]{#1}\loadeddata
    
    % Display the loaded data as a table
    \begin{table}[ht]
    \centering
    \renewcommand{\arraystretch}{0.9}
    \setlength{\tabcolsep}{4pt}
    \pgfplotstabletypeset[
        string type, % Treat columns as strings if needed
        every head row/.style={before row=\hline, after row=\hline}, % Add horizontal lines
        every last row/.style={after row=\hline}, % Add horizontal line after the last row
        columns={Measuring,Fastest,Slowest,Average,Median}, % Specify the columns to display
        display columns/0/.style={column name=Connected Users},
        display columns/1/.style={
            column name=Fastest,
            fixed,
            dec sep align,
            precision=\precision
        },
        display columns/2/.style={
            column name=Slowest,
            fixed,
            dec sep align,
            precision=\precision
        },
        display columns/3/.style={
            column name=Average,
            fixed,
            dec sep align,
            precision=\precision
        },
        display columns/4/.style={
            column name=Median,
            fixed,
            dec sep align,
            precision=\precision
        }
    ]{\loadeddata} % Use the loaded data
    \end{table}

    % Add a bar graph using the loaded data
    \begin{tikzpicture}
        \begin{axis}[
            width=\textwidth,
            height=10cm,
            xlabel={Connected Users},
            ylabel={Time (ms)},
            xtick=data,
            xticklabels from table={\loadeddata}{Measuring},
            legend pos=north west,
            ymajorgrids=true,
            grid style=dashed,
            bar width=3pt, % Set the width of the bars
            ybar,
            legend image code/.code={
                \draw [##1] (0cm,-0.1cm) rectangle (0.6cm,0.1cm);
            },
        ]
    
        \addplot table [x expr=\coordindex, y=Fastest]{\loadeddata}; \addlegendentry{Fastest}
        \addplot table [x expr=\coordindex, y=Slowest]{\loadeddata}; \addlegendentry{Slowest}
        \addplot table [x expr=\coordindex, y=Average]{\loadeddata}; \addlegendentry{Average}
        \addplot table [x expr=\coordindex, y=Median]{\loadeddata}; \addlegendentry{Median}
        \legend{Fastest, Slowest, Average, Median}
    
        \end{axis}
    \end{tikzpicture}
}

% Set PDF document properties
\hypersetup{
    pdfpagelayout   = TwoPageRight,           
    linkbordercolor = {Melon},                
    pdfauthor       = {\authorname},          
    pdftitle        = {\thesistitle},         
    pdfsubject      = {Performance and Correctness Evaluation of Event Broadcasting in the PermaplanT Map Editor}, 
    pdfkeywords     = {event broadcasting, concurrency, Rust, SSE, PermaplanT, performance evaluation, consistency}
}

\setpnumwidth{2.5em}        % Avoid overfull hboxes in the table of contents (see memoir manual).
\setsecnumdepth{subsection} % Enumerate subsections.

\nonzeroparskip             % Create space between paragraphs (optional).
\setlength{\parindent}{0pt} % Remove paragraph indentation (optional).

\makeindex      % Use an optional index.
\makeglossaries % Use an optional glossary.

% Set persons with 4 arguments:
%  {title before name}{name}{title after name}{gender}
%  where both titles are optional (i.e. can be given as empty brackets {}).
\setauthor{}{\authorname}{}{male}
\setadvisor{Univ.Lektor Dipl.-Ing. Dr.techn.}{Markus Raab}{}{male}

\setregnumber{11809490}
\setdate{31}{1}{2025}
\settitle{\thesistitle}{\thesistitlegerman}

\setthesis{bachelor}

% For bachelor and master:
\setcurriculum{Software \& Information Engineering}{Software \& Information Engineering}

\begin{document}

\frontmatter

\addtitlepage{naustrian}
\addtitlepage{english}
\addstatementpage

\begin{acknowledgements*}
I would like to extend my gratitude to the entire PermaplanT team for the excellent collaboration.
It has been a pleasure working with all of them.
My deepest appreciation goes to my professor, Markus Raab, for his guidance, advice and patience throughout this process.
A special thanks to Yvonne Markl for her invaluable Permaculture expertise and positive energy.
Lastly, I am thankful to my wonderful partner, Natashia, for her constant encouragement and for helping me stay balanced during this journey.
\end{acknowledgements*}

\begin{abstract}
PermaplanT is a web application for collaborative garden planning featuring a real-time synchronized Map Editor.
A race condition in the broadcasting mechanism can cause clients to receive actions (change events) in the wrong order, leading to inconsistencies between the state displayed to users and the state stored in the database.
Additionally, PermaplanT should scale to support at least 100 concurrent users without performance degradation.
Using Python script, we conducted two tests: a performance test to measure response times with increasing numbers of concurrent connections, and a correctness test to detect cases where broadcasted actions arrived out of order.
The performance test showed that that the system could handle 100 concurrent users, with increasing response times beyond 1000 connections.
The correctness test confirmed that race conditions can indeed lead to event ordering issues.
We present a solution to resolves the race condition while maintaining scalability, demonstrating the viability of fine-grained locking in asynchronous systems such as PermaplanT.
\end{abstract}

\selectlanguage{english}

% Acronyms go here
\newacronym{tcp}{TCP}{Transmission Control Protocol}
\newacronym{http}{HTTP}{Hypertext Transfer Protocol}
\newacronym{sse}{SSE}{Server Sent Events}
\newacronym{json}{JSON}{JavaScript Object Notation}
\newacronym{uuid}{UUID}{Universally Unique Identifier}
\newacronym{io}{I/O}{Input Output}

\tableofcontents % Starred version, i.e., \tableofcontents*, removes the self-entry.

% Switch to arabic numbering and start the enumeration of chapters in the table of content.
\mainmatter

\chapter{Introduction}\label{chap:introduction}

\section{PermaplanT}

PermaplanT (\url{https://www.permaplant.net}) is an innovative web application that strives to help users plan gardening spaces following the principles of permaculture.
The application uses a web application architecture consisting of a client-side frontend running in the user's web browser and a server-side backend.
The frontend is built using JavaScript and the React framework, while the backend is developed in Rust, backed by the relational database PostgreSQL for data storage.
In addition to the actual application, PermaplanT features a dataset of about 10,000 plant records with information about life cycles, fertility, light and water requirements, and plant relationships.

\section{The Collaborative Map Editor}

The central feature of PermaplanT is its Map Editor, a garden planning tool that allows users to add, remove, and manage garden elements including plants, soil textures, and shading.
To support collaborative garden design, the Map Editor is fully synchronized between users.
Changes made by one user are reflected in real-time to all other users viewing the same map.
This is achieved by sending change events, which we will refer to as \emph{actions}.

\section{Research Objective}

In this thesis, we evaluate the performance and correctness of the action broadcasting mechanism within the PermaplanT Map Editor.
One of PermaplanT's non-functional requirements is to support up to 100 concurrent users making changes to the same map without degradation in response times.
We test whether the current implementation of the action broadcasting mechanism adequately achieves this scalability goal.

In terms of correctness, we will test if the sequence of database updates aligns with the sequence of actions that the client receives.
Some actions are order-dependent, e.g. two actions that move the same planting (an instance of a plant on the map) to an absolute position, or an action that moves a planting and another that deletes the same planting.
If actions arrive out of order, then the state of the database and the state of the user will become inconsistent until the user refreshes the Map Editor page in their browser.

In summary, this thesis answers the following questions:
\begin{enumerate}
  \item \textbf{Performance}: How does the duration of \gls{http} requests that modify the state of a map correlate with the number of concurrent users that receive the broadcasted actions on the same map?
  \item \textbf{Correctness}: Is the order of database writes always the same as the order of actions that the client receives?
\end{enumerate}

\section{Structure of work}

This thesis is structured as follows:

\begin{itemize}[leftmargin=1.5em]  
  \item \textbf{Chapter~\ref{chap:introduction}: Introduction} introduces PermaplanT, its collaborative Map Editor, and defines the research objectives.  
  \item \textbf{Chapter~\ref{chap:background}: Background} explains the technical foundations of PermaplanT’s action broadcasting mechanism and identifies the race condition in its critical section.  
  \item \textbf{Chapter~\ref{chap:related_work}: Related Work} discusses prior approaches to real-time collaboration (e.g. Ethermap, CRDTs) and their relevance to PermaplanT.  
  \item \textbf{Chapter~\ref{chap:methodology}: Methodology} details the proposed locking mechanism to resolve the race condition and describes the performance and correctness tests.  
  \item \textbf{Chapter~\ref{chap:results}: Results} presents empirical data on scalability and event ordering inconsistencies.  
  \item \textbf{Chapter~\ref{chap:discussion}: Discussion} interprets the results, explores alternative solutions, and acknowledges limitations.  
  \item \textbf{Chapter~\ref{chap:conclusion}: Conclusion} summarizes key findings and outlines future work for PermaplanT.  
\end{itemize}

\chapter{Background}\label{chap:background}

\section{Map Updates and State Broadcasting}

When a user opens the Map Editor in their web browser, the frontend sends a \texttt{GET} request to the \gls{http} endpoint at the URL \urlsnippet{/api/updates/maps}.
This establishes a unidirectional stream from the server to the client using \gls{sse}.
Through this stream, the client receives map updates in the format \gls{json} referred to as \emph{actions} within PermaplanT.
Each action has a \texttt{type} field, which defines what information it carries.
For example, actions that affect plantings on the map include the following types:
\begin{itemize}
    \item \texttt{CreatePlanting} creates one or more new plantings.
    \item \texttt{MovePlanting} updates the position (x,y) of one or more plantings on the map.
    \item \texttt{TransformPlanting} changes the size of one or more plantings on the map.
\end{itemize}

For each \gls{http} endpoint that modifies the state of one map, the frontend includes a client-generated \gls{uuid} called \texttt{actionId} in the request body.
The backend validates the request, makes the necessary changes to the PostgreSQL database, and then broadcasts the changes via the appropriate action to all open streams that belong to the same map.
An \texttt{actionId} is generated by the frontend so that the client can recognize and ignore actions that were generated by its own updates.

\subsection{Map Update Example}

To move one or more plantings, the client makes an \gls{http} \texttt{PATCH} request to
\newline \urlsnippet{/api/maps/<map_id>/layers/plants/plantings}
with the desired map identifier and the following \gls{json} payload.

\begin{minipage}{\linewidth}
\begin{lstlisting}[language=json]
{
  "actionId": "<action_id>",
  "dto": {
    "type": "move",
    "content": [
      {
        "id": "<planting_uuid_1>",
        "x": 10,
        "y": 10
      },
      {
        "id": "<planting_uuid_2>",
        "x": 20,
        "y": 20
      }
    ]
  }
}
\end{lstlisting}
\end{minipage}

This request sets the absolute positions of two plantings.

\section{Race Condition in the Broadcasting Mechanism}
\label{sec:race_condition}

All \gls{http} endpoints in the backend follow the steps of the following simplified illustration in Rust.

\begin{minipage}{\linewidth}
\begin{lstlisting}[language=rust]
async update_plantings(
    map_id: i32,
    update: UpdatePlantingDto,
) -> Result<HttpResponse> {
    // Updates the PostgreSQL database
    let updated = plantings::update(map_id, update).await;
    // Broadcast out the state change to all users
    broadcaster.broadcast(map_id, updated).await;
    // Finishes the HTTP request
    return HttpResponse::Ok().json(updated);
}
\end{lstlisting}
\end{minipage}

In asynchronous Rust, each \rustsnippet{async} function is converted into a state machine that stores the intermediate states from which the function can be suspended and resumed at a later time. 
These suspension entry points are denoted by the expressions \rustsnippet{await}.
This allows asynchronous functions to return back control to the runtime, which may continue with the same function or schedule other tasks concurrently\cite{rustasyncdeepdive2024}.

Asynchronous Rust code is best suited \gls{io} heavy applications.
To achieve effective concurrency, runtime libraries employ a small number of operating system threads, each of which can handle many concurrent tasks\cite{rustasyncbookchapter}. 

Many asynchronous programming environments include the runtime as an integrated part of the programming language, e.g. JavaScript in the web browser\cite{mdnjavascripteventloop}.
Rust doesn't provide a default runtime implementation; PermaplanT uses the runtime library Tokio \cite{tokiocrate} to execute asynchronous Rust code.

Since Rust's asynchronous functions give up control, other tasks may run in between two \rustsnippet{await} statements.
When two update requests occur concurrently, their execution can interleave in unpredictable ways, which can cause actions to be broadcasted to the client in the wrong order.
The following scenario illustrates the race condition:
\begin{enumerate}
    \item User 1 makes the \gls{http} request \textit{R1} to move the planting to (100, 100).
    \item User 2 makes the \gls{http} request \textit{R2} to move the same planting to (200, 200).
    \item \rustsnippet{plantings::update().await} finishes in \textit{R1} and stores (100, 100) \
    in the database. The function suspends and the runtime starts executing \textit{R2}.
    \item The runtime fully executes \textit{R2} without interruption.
    (200, 200) is both stored in the database and broadcasted to the client.
    \item Finally, \textit{R1} resumes execution and broadcasts (100, 100).
\end{enumerate}
In this way, the database reflects (200, 200), but the client receives actions this order: (200, 200) and then (100, 100).
This leads to the client incorrectly displaying the planting at the position (100, 100).

\subsection{Critical Section}

As explained by Dubey et al.\cite{artice_race_condition_and_dynamic_data_race_detection}, a race condition occurs when concurrent threads access shared resources without synchronization.
The code segment that accesses the shared resources is called a \textbf{Critical Section}.
In this case, the database update and broadcasting form a critical section together.
Without mutual exclusion, both tasks can occur out of order, leading to inconsistent results.
The article further explains that any solution that deals with securing of critical sections is judged via two mandatory criteria, Mutual Exclusion and Progress.
\begin{enumerate}
    \item \textbf{Mutual Exclusion}
    Only one thread should enter the critical section at any given time.
    In PermaplanT any request that alters the map should finish the entire request, so that requests don't interleave and cause inconsistency.
    In the existing implementation the function \texttt{broadcaster.broadcast()} is secured via one Mutex (Mutual exclusion) that locks the entire broadcaster.
    \gls{sse} guarantees to preserve the order of messages across the network.
    RFC 8895 \cite{rfc8895sse} describes the data stream as a reliable ordered connection between the server and the client.
    
    \item \textbf{Progress}
    Progress ensures that only the threads that execute the critical section should pay for its synchronization cost.
    To achieve this, the synchronization mechanism, such as a Mutex, should be as localized as possible.
    The broadcaster's global Mutex is too broad, hindering performance.
    Instead, the Mutex should only lock the relevant map where the update occurred.
\end{enumerate}

\chapter{Related Work}\label{chap:related_work}

Maintaining consistency and ensuring responsiveness across distributed users are core challenge in collaborative editing systems.
This chapter examines prior work in real-time collaboration tools and concurrency control strategies, focusing on solutions that address event ordering, conflict resolution, and user awareness

Thore Fechner developed a web application called Ehtermap to demonstrate real-time synchronization for collaborative editing of geospatial data on a map\cite{ethermap}.

Ethermap uses continuous synchronization of all changes.
In contrast to PermaplanT's Map Editor, Ethermap displays information about all currently connected users and the area of the map users are currently editing.
To handle conflicts and trace changes, Ethermap provides a continuous history feed of all changes on a map.
Users can inspect revisions and restore previous states.

PermaplanT could benefit from adding visual indicators to signal that other users are working on the same Garden Map.
A history management tool could help users understand changes and experiment more freely, allowing them to revert changes.

Neogy et al. discusses the challenges of maintaining shared awareness among users while supporting independent exploration within a real-time collaboration application. \cite{collaboration_visualizations} 

To bridge that gap, the article introduces interactive features:
\begin{enumerate}
    \item \textbf{Peeking} Temporarily viewing the state of another user.
    \item \textbf{Tracking} Synchronizing views with the views of another user in real time.
    \item \textbf{Forking} Independent exploration starting from the state of another user.
\end{enumerate}

The concept of forking is worth consideration because it allows users to diverge from the shared state and experiment with alternative configurations.
In PermaplanT there is currently no feature that allows users to diverge from the synchronized state of the Garden Map.

Litt et al. presents Peritext \cite{peritext}, a Conflict-Free Replicated Data Type (CRDT) algorithm for collaborative rich text editing.

Its focus is to ensure that edits from multiple users (such as changing text, editing the format, deleting characters) merge with the user's intent in mind.
Rather than storing formatting as a tree (like XML), Peritext assigns a unique identifier to each character and stores formatting as spans linked to characters.
Deleted characters still remain in the text as tombstones.

Permaplant's Map Editor could use ideas from CRDTs and Peritext:
\begin{enumerate}
    \item \textbf{Commutative operations} Peritext employs order-independent operations, where the sequence of user actions does not affect the final state.
    Therefore, there is no need to guarantee the order of events.
    This removes the overhead of synchronization, which is strictly necessary in PermaplanT's implementation.
    \item \textbf{Tombstones} Deleted plantings could remain on the map as hidden tombstones rather than being fully removed.
    This approach would allow PermaplanT to handle actions on deleted plantings gracefully.
    If the deletion is undone, the planting would reappear with all changes made while it was hidden.
    This is not possible in the current system since undoing a deletion creates a new planting (with a new \gls{uuid}).
\end{enumerate}

\chapter{Methodology}\label{chap:methodology}

\section{Testing}

This chapter explains the two types of tests conducted to answer the research questions.
The tests are motivated as follows:
\begin{itemize}
    \item \textbf{Performance Test:} Because the \gls{http} request does not return until the broadcaster finishes, latency may increase with the number of \gls{sse} connections.
    We aim to determine how response times scale as more observers receive updates.
    \item \textbf{Correctness Test:} Given the race condition in the broadcaster, we want to determine how frequently actions are broadcast out of order.
\end{itemize}

\section{Testing Hardware}

The tests were performed on a computer with the following specifications:

\begin{itemize}
  \item \textbf{Operating System} Linux Mint 22 (Linux Kernel 6.8.0-45-generic)
  \item \textbf{CPU} Intel i7-5820K (12) @ 3600MHz
  \item \textbf{RAM} 16 GiB DDR4 @ 2133MHz
  \item \textbf{Storage} 120 GiB Solid State Drive
\end{itemize}

\section{Proposed Solution to the Race Condition}\label{sec:proposed_solution}

To resolve the race condition while preserving performance, we implemented a fine-grained locking mechanism for the broadcaster.
This ensures atomic processing of updates to a specific map, preventing interleaved execution of concurrent requests.

We introduce a \rustsnippet{lock()} function to the broadcaster that grants exclusive read/write access to a single map given the maps identifier.
This function returns a guard that controls the lock's lifetime.
The lock is released when the guard is passed to the \rustsnippet{broadcast} function.
The following simplified Rust code demonstrates the new usage pattern.

\begin{minipage}{\linewidth}
\begin{lstlisting}[language=rust]
async update_plantings(
    map_id: i32,
    update: UpdatePlantingDto,
) -> Result<HttpResponse> {
    // Lock all SSE senders connected to map_id 
    let guard = broadcaster.lock(map_id).await;
    let updated = plantings::update(map_id, update).await;
    broadcaster.broadcast(guard, map_id, updated).await;
    return HttpResponse::Ok().json(updated);
}
\end{lstlisting}
\end{minipage}

\subsection{Implementation Details}

The solution replaces the global \rustsnippet{Mutex} with a concurrent hash map from the \texttt{scc} crate \cite{scccrate}.
Instead of locking the entire broadcaster with a global Mutex, only the relevant map is locked, allowing concurrent updates to other maps.
We chose \texttt{scc} because it provides a concurrent hash map implementation with fine-grained (entry level) locking and concurrent reads.

\section{Performance Test}

To evaluate the performance of the event broadcasting mechanism, we developed the Bash script \bashsnippet{test_perf.sh}.
It measures the response time of \gls{http} requests under varying levels of concurrent SSE connections.
Specifically, we evaluate the slowest, fastest, average, and median response times to understand PermaplanT scales with increasing numbers of concurrent users.
Average and median response times provide the best sense for the systems behavior.

To simulate multiple users viewing the same map, we developed the Python script \texttt{connect\_listener.py}.
The script establishes concurrent connections to the event broadcasting service at the endpoint \urlsnippet{/api/updates}.

The python script \bashsnippet{make_request.py} emulates a user making modifications to the map.
The script begins by creating one planting on the map.
It then sends many \gls{http} \texttt{PATCH} requests in succession that alter the plant's position.
The script outputs the following data points:
\begin{itemize}
    \item the slowest response time
    \item the fastest response time
    \item the average of all response times
    \item the median time
\end{itemize}

\bashsnippet{test_perf.sh} tests the return times for the different amounts of connected users.
We simulate the response times with the following number of connected users: 10, 20, 50, 100, 150, 200, 500, 1000, 2000, 5000 and 10,000.
The numbers of concurrent users  were chosen to represent a wide range of usage scenarios, from light to extreme loads.

For each of these measurements the script follows the following steps:
\begin{enumerate}
    \item Start the PermaplanT backend application, built in release mode to optimize performance.
    \item Sleep 5 seconds to allow the backend to initialize fully and stabilize, minimizing the impact of startup overhead on the measurements.
    \item Simulate the required number of concurrent users by running \newline \bashsnippet{connect\_listener.py}, which establishes connections to the event broadcasting service.
    \item Execute 200 \gls{http} \texttt{PATCH} requests using \bashsnippet{make_request.py}.
    \item Terminate the backend application.
\end{enumerate}

\section{Action Order Test}

The script \bashsnippet{test_order.sh} is designed to test whether database writes and actions occur in the same order.

To capture database writes, we implemented a PostgreSQL migration script that can be added to the backend.
This migration adds a table \texttt{planting\_log}, that keeps a log of all moved plantings.
A trigger is defined to execute before each update to the \texttt{plantings} table, automatically logging the changes into \texttt{planting\_log}.

Additionally, we developed a script called \bashsnippet{attack.py}
to simulate rapid modifications to the same planting via an \gls{http} \texttt{patch} request.
This script is similar to \bashsnippet{make_request.py}, used in the performance tests, but it sends requests in parallel as quickly as possible using Python’s \texttt{threading} library.
The test sends 20,000 requests to move the same planting in parallel.

The script executes the following steps:
\begin{enumerate}
    \item Start the backend in release mode to ensure optimal performance.
    \item Sleep 5 seconds to allow the backend to fully stabilize, minimizing the impact of startup overhead.
    \item Delete all entries in the table \texttt{planting\_log}.
    \item Simulate 200 concurrent users by running \bashsnippet{connection_listener.py}. This creates a high load on the system, increasing the likelihood of encountering the race condition.
    \item Execute \bashsnippet{connect_listener.py} with the \bashsnippet{--record_actions=true} option, which outputs all received actions of type \texttt{MovePlanting}. The output is redirected to the file \texttt{actions.txt}.
    \item Extract and sort the database logs by timestamp into \texttt{db.txt}.
\end{enumerate}

The script produces two text files in the same format:
\begin{enumerate}
    \item The recorded order of actions: \texttt{actions.txt}.
    \item The recorded number of database writes in plantings: \texttt{db.txt}.
\end{enumerate}

We used classic UNIX utility \texttt{diff} \cite{unix-diff-paper} to compare both files line by line.
If we find that the diff order of both files is not the same, we reveal that there is a race condition and events are not sent in the correct order.
However, it is important to note that if the files are identical, it does not guarantee the absence of a race condition.
It simply means that, during this test, the actions and database writes happened to occur in the same order.

\section{Reproducibility}

The tests were conducted against PermaplanT version \texttt{v0.4.3} with the Git hash \newline \texttt{7091b1075}.
All mentioned test scripts have been checked into the Permaplant Git repository under \texttt{benchmarks/sse\_benchmark}.
This includes documentation on how to repeat the tests.

\chapter{Results}\label{chap:results}

\section{Performance Test Results}

\subsection{Current implementation}

The following shows the results of the performance of the current (incorrect) implementation.
The performance test evaluated response times for the existing implementation under concurrent user loads ranging from 10 to 10,000 connections.
All times are in milliseconds.

\perfTableAndGraphFromCsv{perfexisting.csv}

\subsection{Proposed Implementation}

The proposed locking mechanism was tested under identical conditions.
The request times for the suggested implementation of the broadcaster.

\perfTableAndGraphFromCsv{perfsuggestion.csv}

\section{Action Order Test Results}

In one test-run there were 34 action received in the wrong order.
This test clearly confirms that the race condition can lead to actions being sent out-of-order.
The number of out-of-order actions increases when increasing the number of simulations users. 

The proposed improvement from the Section \ref{sec:proposed_solution} showed the same ordering between database writes and actions.

\chapter{Discussion}\label{chap:discussion}

\section{Performance}

Both the current implementation and the suggested fix of the  broadcasters race condition have similar performance characteristics.
Response times remain stable in both implementations according to this test, the application can easily support 100 concurrent users on the same map.
Beyond 100 users, response times increase gradually, with a noticeable spike at 10,000 users.
The steep increase in response times at 10,000 users suggests that the work load does not scale up linearly or that there is a clear bottleneck in the broadcasting mechanism.

\section{Limitations of the performance test}

The performance tests provide a valuable and repeatable way to benchmark the broadcasting mechanism, which can benchmark future improvements to the broadcaster.
These performance tests were conducted in a controlled local environment.
Read-world performance depends on many factors such as network latency, load balancing strategies and other server configuration.
Furthermore, the performance test did further not evaluate how the performance of the frontend gets affected by an increasing number of actions.

\section{Different solutions to the race condition}

The proposed locking of the broadcaster resolves the race condition but it relies on manual implementation discipline.
The developers of PermaplanT have to remember to lock the broadcaster before a critical section when implementing additional requests.
It would be advantageous to make Rust's type system aware of this critical section.
An alternative would be to automatically lock the broadcaster for each request that operates on one map.
This way the developer doesn't have to bother with locking the broadcaster on the correct line.

Chapter~\ref{chap:related_work} highlights Conflict-Free Replicated Data Types (CRDTs) as a potential approach to resolving the race condition.
This would eliminate the need for strict event ordering, as CRDT operations are commutative.
However, CRDTs would require redesigning PermaplanT’s data model and frontend logic, which may outweigh the benefits of order-independent actions.

\chapter{Conclusion}\label{chap:conclusion}

\section{Summary of findings}

This thesis evaluated PermaplanT's Map Editor through two lenses:
\begin{enumerate}
    \item Performance: The system can support the current scalability goal of 100 concurrent users, with response times degrading gracefully until 1,000 users.
    \item Correctness: A race condition in the broadcasting mechanism causes actions to arrive out of order, leading to inconsistencies.
    \item The proposed solution to the race condition in Section \ref{sec:proposed_solution} resolved the race condition while maintaining similar performance characteristics.
\end{enumerate}

\section{Future work}

The following key directions emerge from this thesis:

\begin{itemize}
    \item PermaplanT could adopt a similar solution to the suggested locking of the broadcaster to avoid the race condition.
    Both test scripts can serve the PermaplanT initiative as helpful tools to validate and optimize the broadcaster in a repeatable and automated manner.
    \item Explant performance tests: Given that the Python scripts can be configured to run against the production server, the initiative could build upon the tests to mimic real-world conditions.
    This could include testing how many actions the frontend can process before noticeable performance loss such as dropping the frame-rate.
    \item User experience enhancements: Implement history feeds like Ethermap \cite{ethermap} or conflict visualization inspired by Peritext\cite{peritext} to improve collaborative awareness.
    \item CRDT integration: PermaplanT could explore commutative operations for order-independent actions, reducing synchronization overhead.
\end{itemize}


\backmatter

\bibliographystyle{alphaurl}
\bibliography{thesis}

\end{document}
