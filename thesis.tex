\documentclass[final,draft,oneside]{vutinfth}

% Load packages to allow in- and output of non-ASCII characters.
\usepackage{lmodern}        % Use an extension of the original Computer Modern font to minimize the use of bitmapped letters.
\usepackage[T1]{fontenc}    % Determines font encoding of the output. Font packages have to be included before this line.
\usepackage[utf8]{inputenc} % Determines encoding of the input. All input files have to use UTF8 encoding.

% Extended LaTeX functionality is enables by including packages with \usepackage{...}.
\usepackage[inline]{enumitem} % User control over the layout of lists (itemize, enumerate, description).
\usepackage{multirow}   % Allows table elements to span several rows.
\usepackage{booktabs}   % Improves the typesetting of tables.
\usepackage{subcaption} % Allows the use of subfigures and enables their referencing.
\usepackage{minted}
\usepackage[usenames,dvipsnames,table]{xcolor} % Allows the definition and use of colors. This package has to be included before tikz.
\usepackage{nag}       % Issues warnings when best practices in writing LaTeX documents are violated.
\usepackage{todonotes} % Provides tooltip-like todo notes.
\usepackage{hyperref}  % Enables hyperlinking in the electronic document version. This package has to be included second to last.
\usepackage[acronym,toc]{glossaries} % Enables the generation of glossaries and lists of acronyms. This package has to be included last.
\usepackage{listings, listings-rust} % Code
\lstset{
    basicstyle=\ttfamily,
    keywordstyle=\color{blue},
    commentstyle=\color{gray},
    stringstyle=\color{red},
    breaklines=true
}

% Define convenience functions here:
\newcommand{\authorname}{Jannis Adamek}
\newcommand{\thesistitle}{TODO: title}
\newcommand{\thesistitlegerman}{TODO: title deutsch}

% Set PDF document properties
\hypersetup{
    pdfpagelayout   = TwoPageRight,           % How the document is shown in PDF viewers (optional).
    linkbordercolor = {Melon},                % The color of the borders of boxes around hyperlinks (optional).
    pdfauthor       = {\authorname},          % The author's name in the document properties (optional).
    pdftitle        = {\thesistitle},         % The document's title in the document properties (optional).
    pdfsubject      = {TODO: subject},        % The document's subject in the document properties (optional).
    pdfkeywords     = {TODO, list, of, keywords} % The document's keywords in the document properties (optional).
}

\setpnumwidth{2.5em}        % Avoid overfull hboxes in the table of contents (see memoir manual).
\setsecnumdepth{subsection} % Enumerate subsections.

\nonzeroparskip             % Create space between paragraphs (optional).
\setlength{\parindent}{0pt} % Remove paragraph indentation (optional).

\makeindex      % Use an optional index.
\makeglossaries % Use an optional glossary.

% Set persons with 4 arguments:
%  {title before name}{name}{title after name}{gender}
%  where both titles are optional (i.e. can be given as empty brackets {}).
\setauthor{}{\authorname}{}{male}
\setadvisor{Univ.Lektor Dipl.-Ing. Dr.techn.}{Markus Raab}{}{male}

\setregnumber{11809490}
% TODO: set date
\setdate{27}{10}{2024} % Set date with 3 arguments: {day}{month}{year}.
\settitle{\thesistitle}{\thesistitlegerman}
% \setsubtitle{Optional Subtitle of the Thesis}{Optionaler Untertitel der Arbeit} % Sets English and German version of the subtitle (both can be English or German).

\setthesis{bachelor}

% For bachelor and master:
\setcurriculum{Software \& Information Engineering}{Software \& Information Engineering}

\begin{document}

\frontmatter

\addtitlepage{naustrian}
\addtitlepage{english}
\addstatementpage

\begin{acknowledgements*}

TODO: acknowledgments

\end{acknowledgements*}

\begin{abstract}
    PermaplanT is a web application for collaborative online garden planing. TODO: write abstract
\end{abstract}

\selectlanguage{english}

% Acronyms go here
\newacronym{http}{HTTP}{Hypertext Transfer Protocol}
\newacronym{sse}{SSE}{Server Sent Events}
\newacronym{json}{JSON}{JavaScript Object Notation}
\newacronym{uuid}{UUID}{Universally Unique Identifier}

\tableofcontents % Starred version, i.e., \tableofcontents*, removes the self-entry.

% Switch to arabic numbering and start the enumeration of chapters in the table of content.
\mainmatter

\chapter{Introduction}

\section{PermaplanT}

PermaplanT (\url{https://www.permaplant.net}) is an innovative web application that strives to help users plan gardening spaces following the principles of permaculture.
The application uses a classic web application architecture, consisting of a client-side frontend running in the user's web browser and a server-side backend.
The frontend is built using JavaScript and the React framework, while the backend is developed in Rust, backed by the relational database PostgreSQL for data storage.
In addition to the actual application, PermaplanT features a dataset of about 10,000 plant records with information about life cycles, fertility, light and water requirements, and plant relationships.

\section{The Collaborative Map Editor}

The central feature of PermaplanT is its Map Editor, a garden planning tool that allows users to add, remove, and manage garden elements including plants, soil textures, and shading.
To support collaborative garden design, the Map Editor is fully synchronized, so that changes made by one user are reflected in real time on all clients viewing the same map. This is achieved by sending change events, which we will refer to as \emph{actions}.

\section{Research Objective}

In this thesis, we will evaluate the performance and correctness of the action broadcasting mechanism within the PermaplanT Map Editor.
One of the non-functional requirements of PermaplanT is to support up to 100 concurrent users making changes to the same map without degrading in response times.
We will test whether the current implementation of the action broadcasting mechanism adequately achieves this scalability goal.

In terms of correctness, we will test if the sequence of database updates aligns with the sequence of actions that the client receives.
If actions could arrive out of order and these actions are order-dependent then the state of the client and the state of the database will become inconsistent until the user refreshes the Map Editor page.

In summary, this thesis will awnser:
\begin{enumerate}
  \item \textbf{Performance}: How much longer do requests take to complete relative to how more observers receiving updates?
  \item \textbf{Correctness}: Is the order of actions, that the client reseives, always in the same order as the respective database writes? 
\end{enumerate}

\chapter{Background}

\section{Broadcasting mechanism}

When a user opens the Map Editor in their web browser, the frontend sends a \texttt{GET} request to the \gls{http} endpoint at the url \url{/api/updates/maps}.
This establishes a unidirectonal stream from the server to the client using \gls{sse}.
Through this data stream, the client receives map updates in the \gls{json} format referred to as \emph{actions} within PermaplanT.
Each action has a \texttt{type} field, which defines what information it carries.
For example, actions that affect plantings on the map include the following types:
\begin{itemize}
    \item \texttt{CreatePlanting} creates one or more new plantings.
    \item \texttt{MovePlanting} updates the position (x,y) of one or more plantings on the map.
    \item\texttt{TransformPlanting} changes the size of one or more plantings on the map.
\end{itemize}

For each \gls{http} endpoint that modifies the state of the map, the frontend includes a client-generated \gls{uuid} called \texttt{actionId} in the \gls{json} request body.
The backend validates the request, makes the necessary changes to the PostgreSQL database and then broadcasts the changes via the appropriate action to all open streams to the same map.
\texttt{actionId} is generated by the frontend so that the client can recognize and ignore actions that were caused by its own updates. 

\section{Asynchronous Rust and the Broadcasting Mechanism}

Rust's approach to asynchronous programming is to compile each asynchronous function into a state machine that represents each entry point recursively.
Rust does not provide a standard event loop itself, so we use the popular runtime \texttt{tokio} to supply the event loop and manage asynchronous tasks.

The broadcasting mechanism uses an \texttt{Arc<Mutex<HashMap<i32, ConnectedMaps>>>}, where \texttt{ConnectedMaps} contains a \texttt{Vec<sse::Sender>}.
This structure enables sharing of map connections across multiple threads while ensuring thread safety. 

All \gls{http} updates in the backend follow the steps of this simpified Rust illustration: 

\begin{lstlisting}[language=rust]
{
    // Updates the database
    let updated_planting = plantings::update().await;
    // Send out the update via Server Sent Events
    broadcaster.broadcast(updated_planting).await;
    // Finishes the HTTP request
    return HttpResponse::Ok().json(updated_plantings);
}
\end{lstlisting}

In this process, the broadcaster acquires a lock from the mutex to safely send updates to all connected maps via \gls{sse}. However, because Rust's asynchronous functions give-up control when they encounter an \texttt{await}, other tasks can run while waiting.
This means that another update request might be quick enough to complete its own \texttt{database\_update.await} and queue a broadcast before the previous broadcast has finished.
Thus, there is potential for broadcasts to be out of sync with the database state, particularly if actions are order-dependent.
To illustate the race condition:
\begin{itemize}
    \item User 1 makes a request to move a planting to (100, 100).
    \item User 2 makes a request to move the same planting to (200, 200).
    \item \texttt{plantings::update().await} finishes and stores (100, 100) into the database.
    \item The request of User 2 is quick enough to schedule itself next in the event loop. (200, 200) are both persisted to the database and also sent out via the broadcaster.
    \item Finally (100, 100) is sent out by the broadcaster.
\end{itemize}

This way the final entry in the database is (200, 200) but the actions are sent in this order: (200, 200) and then (100, 100).

The specific motivations for testing are as follows:
\begin{itemize}
    \item \textbf{Performance Motivation:} The \gls{http} request does not return until \texttt{broadcaster.await} concludes, so the latency of each request may increase with the number of \gls{sse} connections.
    We aim to determine how response times scale as more observers receive updates.
    \item \textbf{Correctness Motivation:} Since there is no critical section around the two \texttt{await} calls, an update can complete and another can begin in the middle of the \texttt{database\_update} and \texttt{broadcast} sequence.
    This could cause out-of-order broadcasts, where the \texttt{updated\_data} sent to clients does not align with the current database state.
    If the actions are order-dependent, this misalignment could create inconsistencies between the client and database states.
\end{itemize}

\chapter{Methodology}

\section{Performance Test}

To evaluate the performance of the event broadcasting mechanism in the PermaplanT Map Editor, we conducted a series of performance tests while simulating an increasing number of concurrent users viewing the same map.
These tests measure the response time for user requests.

\subsection{Simulating Connections}
To simulate multiple users viewing the same map, we developed a Python script \texttt{./connect\_listener.py}.
\texttt{./connect\_listener.py} establishes concurrent connections to the event broadcasting service at the endpoint \texttt{/api/updates}.
The number of connections and the nature of each connection can be configured through command-line arguments.
We simulated 100 concurrent users on map with the ID of 1 on localhost by executing the following command: 
\begin{lstlisting}[language=sh]
python3 ./connect\_listener.py --url localhost --map\_id 1 100
\end{lstlisting}
This initiates 100 Python threads (green threads), each of which opens an \gls{sse} connection.  

\subsection{Simulating User Interaction with the Map}
A second Python script emulates a user making modifications to the map.
The script begins by creating a planting on the map and then continuously alters its position by sending an \gls{http} \texttt{PATCH} request to \texttt{api/maps/<MAP\_ID>/layers/plants/plantings}.

It can be configured how many request the script makes.
In our benchmark we used the invocation \texttt{python3 make\_requests.py --url localhost --map\_id 1 100} to make
100 move requests - it delete the slowest and the fastest measurements and returns the average time each request takes to complete.

\subsection{Tests}

Each test is conducted by connecting a number of concurrent users while measuing the request times using the second script. The tests are run both locally and against the production server at \texttt{https://www.permaplant.net}.

\section{Local tests}

The local tests are run on a computer with the following specificaitons:
\begin{itemize}
  \item Operating System: Linux Mint 22 (Linux Kernal 6.8.0-45-generic)
  \item CPU: Intel i7-5820K (12) @ 3600MHz
  \item RAM: 16 GiB DDR4 @ 2133MHz
  \item Storage: 120 GiB Solid State Drive
\end{itemize}

Each test follows the following steps:
\begin{enumerate}
    \item start the backend (release build)
    \item use the python script to emulate connected users
    E.g \texttt{./connect\_listener.py --map\_id 1 10} for 10 connections
    \item In a different shell execute 200 requests \texttt{./make\_request.py 200}
\end{enumerate}

In order to try 2000 concurrent \gls{sse} connections locally we had to adjust the number of maximum open file descriptors using the \texttt{ulimit} command.
The default for this limit on Linux Mint 22 is 1024 (shown by \texttt{ulimit -n}).
We adjusted this limit to 3000 before simulating 2000 concurrent \gls{sse} connection.
Without this adjustment Backend would log \texttt{Too many open files (os error 24)}.

\section{Test against www.permaplant.net}

Our production server is hosted on www.permaplant.net.
On one computer we use the connection script while a different computer on a different network makes the requests.

On the above mentioned computer we simulate our connected useres executing: 
\begin{verbatim}
# For 10 connections
./connect\_listener.py \
    --url https://www.permaplant.net \
    --map\_id 79 \
    --client\_id "PermaplanT-Prod" \
    10 
\end{verbatim}

On a different computer (Thinkpad T470 i5-6300U) in a different network we simulate the requests:
\texttt{./make\_request.py --url https://www.permaplant.net --map\_id 79 --client\_id "PermaplanT-Prod" 200}

\section{Action Order Test}

This test investigates whether actions may reach the client in a different order than the sequence in which their corresponding changes are committed to the database.
To record planting position updates we added new table called \texttt{planting\_log} into our PostgreSQL database, that keeps a log of all moved plantings, with their old and new x and y coordinates.

\begin{lstlisting}[language=sql]
CREATE TABLE IF NOT EXISTS planting_log (
    log_timestamp timestamptz DEFAULT clock_timestamp(),
    planting_id uuid NOT NULL,
    old_x int,
    old_y int,
    new_x int, kkm
    new_y int,
    PRIMARY KEY (log_timestamp, planting_id)
);
\end{lstlisting}

A Trigger logs every update to the \texttt{plantings} table into \texttt{planting\_log}.

\begin{lstlisting}[language=sql]
CREATE TABLE IF NOT EXISTS planting_log (
    log_timestamp timestamptz DEFAULT clock_timestamp(),
    planting_id uuid NOT NULL,
    old_x int,
    old_y int,
    new_x int,
    new_y int,
    PRIMARY KEY (log_timestamp, planting_id)
);

CREATE OR REPLACE FUNCTION log_planting_update()
RETURNS trigger AS $$
BEGIN
    INSERT INTO planting_log (log_timestamp, planting_id, old_x, old_y, new_x, new_y)
    VALUES (clock_timestamp(), OLD.id, OLD.x, OLD.y, NEW.x, NEW.y);
    RETURN NEW;
END;
$$ LANGUAGE plpgsql;

CREATE TRIGGER log_planting_update_trigger
BEFORE UPDATE ON plantings
FOR EACH ROW
EXECUTE FUNCTION log_planting_update();
\end{lstlisting}

Using a similar connection script from the performance test we recorded all actions into a file. \texttt{./connect\_listener.py 1 1> ./target/actions.txt}

We develped a script called \texttt{attack.py} that move a planting via the \gls{http} \texttt{patch} request.
Lastly we extracted the database planting\_logs into a file with the same format as \texttt{actions.txt}.
Each line is in the format planting\_id | x | y.
We used the \texttt{diff} program to compare both files line by line.
If we find that the order is not the same we reveal that there is a race condition and events are not sent in the correct order. 
Note that the oposite is not the case: If the order of both files are the same the broadcasting system might not have race conditions or the test may be inadequat of showing it.

\chapter{Results}

The results clearly show that the current implementation of the broadcasting system can support 100 concurrent users.

\section{Performance Test}
\subsection{Local tests}

\begin{table}[h!]
    \centering
    \begin{tabular}{cc}
        \toprule
        \textbf{Connections} & \textbf{Avg Time (200 Requests) in ms} \\
        \midrule
        0 & 6.39 \\
        10 & 6.20 \\
        20 & 6.66 \\
        50 & 6.50 \\
        100 & 6.24 \\
        150 & 6.86 \\
        200 & 7.58 \\
        500 & 9.85 \\
        1000 & 10.97 \\
        2000 & 10.10 \\
        \bottomrule
    \end{tabular}
    \caption{Results of 200 Requests for Different Connection Counts}
\end{table}

An interesting note: When simulating 2000 concurrent connections the backend startet to report \texttt{Too many open files (os error 24)}.
This is because the default for the the maximum number of open file descriptors per process on Linuxmint 22 is 1024.
This can be found out by executing \texttt{ulimit -n}.


\subsection{Test against www.permaplant.net}

\begin{table}[h!]
    \centering
    \begin{tabular}{cc}
        \toprule
        \textbf{Connections} & \textbf{Avg Time (200 Requests) in ms} \\
        \midrule
        0 & 64.06 \\
        10 & 60.24 \\
        20 & 60.81 \\
        50 & 61.52 \\
        100 & 64.76 \\
        150 & 64.55 \\
        200 & 62.42 \\
        \bottomrule
    \end{tabular}
    \caption{Results of 200 Requests for Remote Test}
\end{table}

\section{Action Order Test}

\chapter{Discussion}

\section{Performance Test}

\section{Action Order Test}

\printglossary[type=\acronymtype,title=Acronyms]

\backmatter

\bibliographystyle{alpha}
\bibliography{thesis}

\end{document}
